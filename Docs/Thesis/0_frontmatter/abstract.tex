
% Thesis Abstract -----------------------------------------------------


%\begin{abstractslong}    %uncommenting this line, gives a different abstract heading
\begin{abstracts}        %this creates the heading for the abstract page

Resource management is no doubt one of the key problems that all clusters have to face. 
The LOFAR telescopes observe the sky and archive the records. 
Though there are computation facilities to process the observation data, the quantity of data is far beyond their capability. 
Therefore, the data is fetched and processed through a multiple-step pipeline when it is needed. 
Each step may take a long time and consume a significant amount of computation power. 
Currently, we have horizontally scalable implementations in MPI and Spark。
The computation power is positively correlated to the numbers of involved computing units.
However, both of them have an intrinsic drawback on resource utilizing. 
To promote the utilization of the resources, we propose an auto-provisioning distributed computing system. 
The auto-scaling mechanism enables the applications to dynamically fetch and release resources, and as the consequence,  the resources of the cluster are used to the maximum extent. 
The results show that the nominal resource utilization of a cluster can be improved up to 99.9\%. 
With idle resources being used, users may take less time to wait. 
In our busy cluster scenario test case, users take 10\% less time on average to wait for the job to be finished at the submission of the jobs.


\end{abstracts}
%\end{abstractlongs}


% ---------------------------------------------------------------------- 
