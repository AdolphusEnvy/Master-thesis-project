% this file is called up by thesis.tex
% content in this file will be fed into the main document

\chapter{Experiments, results and analysis } % top level followed by section, subsection


% ----------------------- paths to graphics ------------------------

% change according to folder and file names
\ifpdf
    \graphicspath{{4_analysis&results/figures/PNG/}{4_analysis&results/figures/PDF/}{4_analysis&results/figures/}}
\else
    \graphicspath{{4_analysis&results/figures/EPS/}{4_analysis&results/figures/}}
\fi


% ----------------------- contents from here ------------------------
% 
In this chapter, we will list and analyze the results of a few experiments for performance testing. 
The performance of this system includes the efficiency of scheduling algorithm which can be measured by resource utilization of the cluster, and the efficiency of parallel computation among executors, which can be measured by drawing its speedup line and comparing it with the theoretical speedup.

In the following sections, first, the use case for testing this system will be explained, which is a calibration use case on LOFAR data set.
After that, we will show how a data processing job from users can be accelerated via adding up more computation resources.
In the last part, we will test the overall performance of this system by comparing the resource utilization of cluster and the user waiting time between the cases to which we introduce this system or not.
The experiments are all performed on the DAS-5 Leiden site which contains 24 computation nodes, and each has dual 8-core CPU with 2.4GHz speed and 64 GB memory.

%-----------------------------overview-----------------------------------
\section{Sagecal calibration use case}
To test this system, we setup our test case to process and calibrate the data collected by the LOFAR telescope. 
As a radio astronomical calibration package\footnote{\url{http://sagecal.sourceforge.net/}}, SAGECal (Space Alternating Generalized Expectation Maximization Calibration) is fast, distributed, and GPU accelerated.
It firstly solves the problem that most original solutions are direction-dependent, which means that the algorithm cannot be horizontally scaled.
And then, it is extended by MPI for horizontal scaling, and GPU for vertical acceleration.

First, the paths to the dataset, sky model, and output file should be given.
There are plenty of parameters, and the most critical parameter that we should concern in this system for resource utilization is the number of threads.
Normally, it will be configured as the number of (physical)cores in the node where the process is executed due to the fact that hyper-thread is not helpful for Sagecal.
This number can be fetched from runtime parameters of JVM, and will return the logic core.
After that, the number divided by 2 gives the number of physical cores, which makes it flexible to any environment.
However, as mentioned before, there is a side executor alongside the master.
To prevent Sagecal from exhausting all the computation ability, the number of threads is configured by subtracting 2 from the  core number.

The Sagecal is well-encapsulated, and a typical use on single node can be done by  the command line below.

\begin{verbatim}
  $ sagecal -d myData.MS -s mySkymodel -c myClustering -n no.of.threads \
            -t 60 -p mySolutions -e 3 -g 2 -l 10 -m 7 -w 1 -b 1
\end{verbatim}
Due to the fact that the Sagecal and related environment are packed in a Singularity container, an example for computing is shown below.
\begin{verbatim}
  $ singularity exec Sagecal.simg /opt/sagecal/bin/sagecal PARAMETERS
\end{verbatim}
Therefore, the workers execute task by creating new process and running command line above. 
And the main thread of workers will be blocked until the task is finished.

\section{Distributed parallel computation}

For distributed parallel computation, the only and most important key metric lies in the acceleration by the increase of the computation resource.
A test data set is created by duplicating a sample data set 150 times as sub-data sets. 
For each sub-data set, the size is 67 MB, and it takes around 20 seconds for computation.


\begin{table}[]
    \centering
    \begin{tabular}{ccccc}
    \hline
    Num of Nodes      & 1         & 2         & 4       & 8       \\ \hline
    Time consumption(ms)  & 3,584,484 & 1,772,311 & 890,327 & 458,169 \\
    Acceleration      & 1         & 2.022     & 4.026   & 7.823   \\
    Theoretical speed-up & 1         & 2.333     & 5.000   & 10.333  \\
    Efficiency rate   & 100\%     & 86.7\%    & 80.5\%  & 75.7\% 
    \end{tabular}
    \caption{Time consumption by the different number of node}
    \label{tab:acc}
\end{table}

Therefore, we performed four experiments, and record the time spent on the same test by different number of nodes. The results are shown in Table.  \ref{tab:acc}. To make the demonstration  more intuitive, the speed-up is visualized, as shown in Fig.\ref{4_analysis&results/figures/acc.png}.
\figuremacroW{4_analysis&results/figures/acc.png}{Performance of computation layer with different number of nodes}{The ideal speed-up should follow $y=\frac{8}{6}x-\frac{1}{3}$ instead of $y=x$}{1}

In most cases, the theoretical speed-up should follow the linear one.
However, it can be seen from the results that the speed-up is super-linear.
The reason is that the baseline here is the case with one master, and one side executor.
As mentioned before, the side executor uses the maximum number which subtracts two cores.
Therefore, a new node shows 8/6 $\approx$ 1.33 times of computation power compared with the side executor.  
The theoretical linear speed-up should follow the line  $y=\frac{8}{6}x-\frac{1}{3}$.
Thus, we consider this line as the upper-bound of the theoretical speed-up.

After the adjustment of the theoretical linear speed-up, the problem of inefficiency is revealed.
It can be seen from Table. \ref{tab:acc} that, by calculating the ratio between the real acceleration and theoretical speed-up, the efficiency keeps dropping with the increase of the resource introduced.
According to the detailed performance metrics, the huge performance lost is due to the frequent connection build-up. 
In the implementation, the connection between ports is configured as an exclusive one-to-one connection.
This means that every time the master sends a task to an idle worker, it has to disconnect to the previous worker’s port, and connect to a new worker.
This overhead cost is around one second per time.
According to Amdahl’s law, with the increase of acceleration, this part of the time cost becomes increasingly larger.
In this experiment setting, for the master, it will take 150 seconds for constant connecting, sending, and disconnecting.
Of course, if the entire job requires hours to be processed, the connection overhead will have fewer effects on the performance.
Another optimization is to increase the batch size, which reduces the number of tasks and connections.


Besides the computation performance, the fault tolerance features are also tested.
The crash of either workers or the master will not result in the crash of the entire system.
The jobs will be finished eventually as long as not all the resources are released.

\section{Resource utilization optimization}
For the user of this system, resource utilization is the key metric as promised.
And of course, the processing speed is vital as well. In this section, the overall performance of the system is tested.
There are two key metrics that can be observed and considered for performance, as expressed below.
\begin{itemize}
    \item Nominal utilization: $A/T$; for cluster, it measures the resource utilization.
    \item User waiting time: $FinishTime-SubmitTime$; for calibration users, it represents the time they should wait during the processing.
\end{itemize}
The nominal utilization is called \textit{Nominal} because it is not able to reflect the real utilization.
For example, when there is no calibration job on the run, the system still tries to make the resources fully used.
Besides the nominal utilization, the average resource usages of each kind of job are also monitored.
This system should not be harmful to other users’ jobs, which means by deploying this system, the average resource usage of other jobs should not drop too much.

\subsection{Simulation settings}
To test the system’s performance, and measure its efficiency, a reproducible simulation procedure is implemented.
It should simulate the submission and process of different kinds of jobs under both the traditional environment and the environment where this system is introduced.
The simulation should provide a fair comparison for this system, and the job submission should be repeatable in the case either with or without it.

Therefore, a simulation program is created by loading a $submitList$, which lists the normal jobs and calibration jobs, and submits those jobs at a specific time with corresponding configuration. 
Each line in the $submitList$ file contains the number of nodes, type of job(named \textit{normal/calibration}), time limit, real running time, submission time(from the start of the simulation), and parameters.
The simulation has two modes, that is, MPI mode where all jobs are considered as a batch job submitted by SLURM interface; and the scale mode where the resource manager is on, and the calibration jobs will be submitted to the web service instead of SLURM, besides, the reset is the same.
According to the submission time, the jobs are sent by the configuration. 
And the simulation will end 30 minutes after the submission of the last job on the list.

The calibration jobs in MPI mode are not real processing, instead, they are the same as normal jobs, i.e., executing $sleep$ command with a certain time.
With the configuration, the calibration jobs in MPI mode will take five nodes for 240 minutes, which is close to the real case that executes test data set by five nodes.

Besides, there is a monitor process which records the status of the cluster.
The monitor keeps recording the resource occupation of different jobs (e.g., calibration, normal, others, and pending).
The simulation program records the submission of the calibration jobs, and the calibration application itself logs the finishing of calibration jobs.
Thus, the waiting time can be calculated.
In addition to these types of data, the calibration application will keep recording the $miniNodes$ and the number of left tasks by the time.

\subsection{Non-busy case simulation }
First, we consider the scenario of the Non-busy cluster. In this case, most of the time, the cluster is not fully utilized due to a lack of jobs.
The simulation is described in \ref{submit:1}, which consists of 19 jobs, including four calibration jobs and 15 normal jobs as the submission of other users. 
With the submitList, the simulation will take around 11 hours, and in the middle, there will be a lot of idle nodes for a few hours in MPI mode. 
The resource utilization of different kinds of jobs is illustrated in \ref{fig:nonBusyMPIUTI}. The cluster has 24 working nodes, and during the experiments, the average of resource occupation is 19.79(nodes), which means the overall resource utilization rate is 82.47\%.
In detail, the calibration jobs take 5.81 nodes on average, and normal users take 13.99 nodes.
\begin{figure}
    \centering
    \begin{minipage}{.48\textwidth}
      \centering
      \includegraphics[width=1\linewidth]{4_analysis&results/figures/MPIUtiNonBusy.png}
      \caption[Resource utilization on MPI mode ,non busy case]{{\small\textbf{Resource utilization on MPI mode ,non busy case} - The overall resource utilization is 82.47\%}}
      \label{fig:nonBusyMPIUTI}
    \end{minipage} 
    \begin{minipage}{.48\textwidth}
      \centering
      \includegraphics[width=1\linewidth]{4_analysis&results/figures/MPIGanttNonBusy.png}
      \caption[Gantt chart of calibration jobs]{{\small\textbf{Gantt chart of calibration jobs} - Benefit from adequate resources jobs start once they are submitted, the vertical line is the time simulation ends }}
      \label{fig:nonBusyMPIgantt}
    \end{minipage}
\end{figure}
%\figuremacroW{4_analysis&results/figures/MPIUtiNonBusy.png}{Resource utilization on MPI mode ,non busy case}{The overall resource utilization is 82.47\%}{0.85}
And in Fig. \ref{fig:nonBusyMPIgantt},the Gantt chart shows that the waiting duration approximates to the running time of each job due to the fact that all jobs almost start immediately.
%\figuremacroW{4_analysis&results/figures/MPIGanttNonBusy.png}{Gantt chart of calibration jobs}{ Benefit from adequate resources jobs start once they are submitted}{0.85}
\begin{figure}
    \centering
    \begin{minipage}{.48\textwidth}
      \centering
      \includegraphics[width=1\linewidth]{4_analysis&results/figures/ScaleUtiNonBusy.png}
      \caption[Resource utilization after introducing this system ,non busy case]{{\small\textbf{Resource utilization after introducing this system ,non busy case} - The overall resource utilization is 99.83\%}}
      \label{fig:nonBusyScaleUTI}
    \end{minipage} 
    \begin{minipage}{.48\textwidth}
      \centering
      \includegraphics[width=1\linewidth]{4_analysis&results/figures/ScaleGanttNonBusy.png}
      \caption[Gantt chart of calibration jobs]{{\small\textbf{Gantt chart of calibration jobs} - Job 47 is accelerated due to extra resources }}
      \label{fig:nonBusyScalegantt}
    \end{minipage}
\end{figure}
And then, we perform simulation with our system and the same submitList. From Fig. \ref{fig:nonBusyScaleUTI}, we can observe the space under the horizontal line, which indicates that the maximum resource is almost filled.
The nominal overall resource utilization is  99.83\%.The average resource occupation of normal jobs is 13.97 nodes, and for calibration application, it increases to 9.99 nodes.
Taking into consideration the waiting time, Fig. \ref{fig:nonBusyScalegantt} shows that automatic scaling policy calibration jobs can be accelerated based on to the extra resources.
The waiting time is shortened from 240 minutes to 121 minutes. Besides the resource occupation, the change of $miniNodes$, and the number of left tasks is displayed in Fig. \ref{fig:nonBusyScaleUTI} as well.
The relation between $miniNodes$ and $leftTasks$ can be formulized as Eq. \eqref{con:mininode}.
\begin{equation}
    miniNodes=\left\{
        \begin{aligned}
        2 &      & {leftTasks      <      5}\\
        4 &      & {5< leftTasks      <      20}\\
        leftTasks/20+3 &      & {20< leftTasks      <      100}\\
        10 &      & {else}\\
        \end{aligned}
        \right.\label{con:mininode}
\end{equation}
Please be noted that the waiting time reduction of Job 45 is not due to the auto-scaling, instead, it is because of the uniformed job entrance and the job queue.
The uniformed job queue follows the FCFS order, besides, with the resources for two jobs, the first job can take all the resources occupied by this system, while the second one has to wait. 
Once the first job is done, the second job may be able to take double resources, and finished as expected.

This simulation shows that their jobs can be accelerated for the users of the calibration application when the cluster is not busy.
At the same time, other users will not have a bad effect.

\subsection{Busy case simulation}

In a non-busy scenario, the overall resource utilization is not high in nature for normal batch cluster.
This system initially aims to solve the problems of the limitations of the backfilling scheduling policy.
Therefore, we should make a simulation with a large number of jobs to investigate how resource utilization can be improved on top of backfilling optimization. 
The \ref{submit:2} describes the scenario of busy cluster. There will be 24 jobs submitted to the cluster/system which includes 8 calibration jobs and 16 normal jobs.
\begin{figure}
    \centering
    \begin{minipage}{.48\textwidth}
      \centering
      \includegraphics[width=1\linewidth]{4_analysis&results/figures/MPIUtiBusy.png}
      \caption[Resource utilization on MPI mode ,busy case]{{\small\textbf{Resource utilization on MPI mode ,busy case} - The overall resource utilization is 90.57\%}}
      \label{fig:BusyMPIUTI}
    \end{minipage} 
    \begin{minipage}{.48\textwidth}
      \centering
      \includegraphics[width=1\linewidth]{4_analysis&results/figures/MPIGanttBusy.png}
      \caption[Gantt chart of calibration jobs]{{\small\textbf{Gantt chart of calibration jobs} - Jobs need to take a while waiting for resources}}
      \label{fig:BusyMPIgantt}
    \end{minipage}
\end{figure}
It can be seen from Fig. \ref{fig:BusyMPIUTI}, the utilization can be measured, and on average 90.57\% of the resources are occupied.
However, sometimes, part of the resources is idle due to the limit of scheduling policy, even though there are plenty of jobs on the pending.
Fig. \ref{fig:BusyMPIgantt} shows that users have to wait longer for finishing the calibration jobs compared with the non-busy case.




\begin{figure}
    \centering
    \begin{minipage}{.48\textwidth}
      \centering
      \includegraphics[width=1\linewidth]{4_analysis&results/figures/ScaleUtiBusy.png}
      \caption[Resource utilization after introducing this system ,busy case]{{\small\textbf{Resource utilization after introducing this system ,busy case} - The overall resource utilization is 99.86\%}}
      \label{fig:BusyScaleUTI}
    \end{minipage} 
    \begin{minipage}{.48\textwidth}
      \centering
      \includegraphics[width=1\linewidth]{4_analysis&results/figures/ScaleGanttBusy.png}
      \caption[Gantt chart of calibration jobs]{{\small\textbf{Gantt chart of calibration jobs} - Users need less time for waiting}}
      \label{fig:BusyScalegantt}
    \end{minipage}
\end{figure}
\begin{table}[]
    \centering
    \begin{tabular}{ccccccccc}
    \hline
    Job ID         & 0      & 1      & 2      & 3      & 4      & 5      & 6      & 7     \\
    \hline
    MPI mode(Min)   & 240.01 & 240.01 & 402.60 & 335.53 & 414.91 & 316.52 & 206.51 & 49.17 \\
    Scale mode(Min) & 118.15 & 251.99 & 191.73 & 328.04 & 332.95 & 281.57 & 206.51 & 49.18
    \end{tabular}
    \caption{Waiting time of jobs comparison on MPI mode and Scale mode}
    \label{tab:waittimecomparison}
\end{table}
After the introduction of this system, the overall nominal utilization climbs to 99.86\%. 
In Fig. \ref{fig:BusyScaleUTI}, the curve of the left task maintains a high position, which means that even though there will be resource waste during the down-scaling stage, the resources allocated to calibration application are always fully used compared with the non-busy case.
In this experiment, the average resource occupation by calibration jobs increases from 10.44 nodes to 11.88 nodes, and for other users, the average resource occupation is also increased from 11.30 nodes to 12.08 nodes.
Thus, all users in the cluster benefit from this system.

In detail, for the calibration applications, by comparing the waiting time of calibration jobs, the extra resources also benefit the calibration jobs.
It can be observed from  Table. \ref{tab:waittimecomparison}that, except for the jobs of 0 and 1, they are affected by the queue strategy and the jobs of 6 and 7, which are not finished at the end of the simulation. 
All other jobs, the jobs from 2-5, take less time after the introduction of this system.

\subsection{Resource utilization testing on real environment}
The simulations above are constructed manually.
To evaluate the performance of resource management in real scenarios, a 7-day experiment was performed, which started at 0:00 AM 22ed of August and ends at 11.59 PM 28th of August in 2020.
The experiment was performed in the DAS5-Leiden site with full 23 available working nodes. 
The resource occupation is displayed in Fig. \ref{4_analysis&results/figures/7days.png}  
\figuremacroW{4_analysis&results/figures/7days.png}{7 days experiment with other users come and go}{The overall resource utilization is 99.99\%}{1}
Unfortunately, due to the summer session, only two other users submitted the jobs during the experiment time.
The overall nominal resource utilization achieves 99.99\%.
% ---------------------------------------------------------------------------
% ----------------------- end of thesis sub-document ------------------------
% ---------------------------------------------------------------------------