% this file is called up by thesis.tex
% content in this file will be fed into the main document

\chapter{Experiments, results and analysis } % top level followed by section, subsection


% ----------------------- paths to graphics ------------------------

% change according to folder and file names
\ifpdf
    \graphicspath{{4_analysis&results/figures/PNG/}{4_analysis&results/figures/PDF/}{4_analysis&results/figures/}}
\else
    \graphicspath{{4_analysis&results/figures/EPS/}{4_analysis&results/figures/}}
\fi


% ----------------------- contents from here ------------------------
% 
In this chapter, we describe and analyze the results of a few experiments we have done to test the performance of the proposed system. 
The performance of this system includes
\begin{itemize}
  \item The efficiency of scheduling algorithm - measured by resource utilization of the experiment cluster.
  \item The efficiency of parallel computation among executors - measured by drawing its speedup line and comparing it with the theoretical speedup.
\end{itemize}

In the following sections, first, we describe the test scenarios we have used to evaluate the performance of the proposed system.
After that, we show how data processing jobs can be accelerated by adding more computation resources.
In the last part, we test the overall performance of the proposed system by comparing the resource utilization of the cluster and the user waiting time before and after we introduced the proposed user-level resource scheduling.
All experiments are performed on the DAS-5 Leiden site which contains 24 computation nodes, and each has dual 8-core CPUs with 2.4GHz speed and 64 GB memory.

%-----------------------------overview-----------------------------------
\section{SAGECal calibration use case}
To test the proposed user-level resource management system, we use the calibration data pipeline of the LOFAR to process and calibrate the data collected by the LOFAR telescope. 
The latest radio astronomical calibration package in use is SAGECal (Space Alternating Generalized Expectation Maximization Calibration)\footnote{\url{http://sagecal.sourceforge.net/}}.
It is fast and enabled with distribution and GPU acceleration features.
In the test set, all worker nodes exploit SAGECal to process data in parallel by given configuration.

To run SAGECal we need to provide a number of parameters, we discuss here only the one that is relevant to our tests, namely the number of threads.
The number of threads is equal to the physical cores, it is based on the logic core number collected from the JVM runtime.  
However, as mentioned before in Section \ref{actionOfExcution}, there is a side executor alongside the master to maximize the resource utilization.
To prevent the side executor from exhausting all the computation ability of the physical machine shared with the master executor, the number of threads for this worker executor is configured by subtracting 2 from the physical core number.

The SAGECal is well-encapsulated, and a typical use on a single node can be done by the command line below.

\begin{verbatim}
  $ sagecal -d myData.MS -s mySkymodel -c myClustering -n no.of.threads \
            -t 60 -p mySolutions -e 3 -g 2 -l 10 -m 7 -w 1 -b 1
\end{verbatim}
SAGECal is deployed as singularity container, and can be run as follows:
\begin{verbatim}
  $ singularity exec Sagecal.simg /opt/sagecal/bin/sagecal PARAMETERS
\end{verbatim}

The way worker executors exploit SAGECal to process data is to create a new process inside the Java program and use system call to launch data processing.


\section{Distributed parallel computation}
A test data set is created by duplicating a sample data set 150 times and each of them is a sub-data set. 
For each sub-data set, the size is 67 MB, and it takes about 20 seconds for computation.


\begin{table}[]
    \centering
    \begin{tabular}{ccccc}
    \hline
    Num of Nodes      & 1         & 2         & 4       & 8       \\ \hline
    Time consumption(ms)  & 3,584,484 & 1,772,311 & 890,327 & 458,169 \\
    Acceleration      & 1         & 2.022     & 4.026   & 7.823   \\
    Theoretical speed-up & 1         & 2.333     & 5.000   & 10.333  \\
    Efficiency rate   & 100\%     & 86.7\%    & 80.5\%  & 75.7\% 
    \end{tabular}
    \caption{Execution time by the different number of node}
    \label{tab:acc}
\end{table}

We performed four experiments and recorded the execution time of each test in the function of the number of computing nodes.
 The results are shown in Table.  \ref{tab:acc} and the speed-up is visualized in Fig.\ref{4_analysis&results/figures/acc.png}.
\figuremacroW{4_analysis&results/figures/acc.png}{Performance of computation layer with different number of nodes}{The ideal speed-up should follow $y=\frac{8}{6}x-\frac{1}{3}$ instead of $y=x$}{1}

In most cases, the curve of theoretical speed-up should be lower than that of linear speedup.
However, as it can be seen from Fig. \ref{4_analysis&results/figures/acc.png}, the speed-up is super-linear.
This super-linear speed-up originates from the inequality of computation speed between the side executor and the following additional worker executors.
When there is only one physical node assigned to the system, it only uses full minus 2 cores to process data.
In an 8-core cluster, every new involved physical node contributes 8/6 $\approx$ 1.33 times of computation power to the system in theory.
Therefore, an adjusted theoretical linear speed-up should follow the line  $y=\frac{8}{6}x-\frac{1}{3}$.

After the adjustment of the theoretical linear speed-up, the problem of inefficiency is revealed.
It can be seen from Table. \ref{tab:acc} that, by calculating the ratio between the real acceleration and theoretical speed-up, the efficiency keeps dropping with the increase of the resource introduced.
According to the detailed performance metrics, the performance loss comes from connection build-up. 
In the implementation, the connection between ports is configured as an exclusive one-to-one connection.
This means that every time the master sends a task to an idle worker, it has to disconnect to the previous worker’s port and connect to a new worker.
This overhead cost is around one second per connection.
In the experiment setting, with 150 tasks, connection takes 150 seconds for constant connecting, sending, and disconnecting.
Of course, For long executions (hours) this overhead can be negligible 
Another optimization is to increase the batch size, which reduces the number of tasks and connections.


Besides the computation performance, we also tested the fault tolerance feature.
The crash of either workers or the master will not result in the crash of the entire system.
The jobs will be finished eventually as long as not all the resources are released.
The dynamic scaling of the system depends on the fault tolerance mechanism.
The completion of jobs in a dynamic scaling setting can support the conclusion that the fault tolerance feature is enabled.

\section{Resource utilization optimization}
For the user of this system, resource utilization is the key metric.
In this section, the overall performance of the system is tested.
We observe and consider two key metrics for performance.
\begin{itemize}
    \item Nominal utilization: $A/T$; for cluster, it measures the resource utilization.
    \item User waiting time: $FinishTime-SubmitTime$; for calibration users, it represents the time they should wait.
\end{itemize}
The nominal utilization is called \textit{Nominal} because it does not reflect the real utilization.
For example, when there is no calibration job running, the system still tries to fully use the resource of the system.
Besides the nominal utilization, the average resource usages of each kind of job are also monitored.
One of the key features of the proposed use level resource management is not to be greedy and lead to being harmful to other users’ jobs, which means the average resource usage of other jobs should not drop too much.

\subsection{Simulation settings}
To test the system’s performance, and measure its efficiency, a program to simulate various loads on the cluster has been implemented.
We wrote an application that simulates multi-user jobs submission.

The application submits jobs including SAGECal jobs at various times to simulate the different load of the clusters.
Each job submission specifies the time limit, real running time, submission time (from the
start of the simulation), and other job-specific parameters.
The application has two modes: SLURM-only mode where all jobs are considered as a batch job submitted by SLURM interface; and the scale mode where the resource manager is on, and the calibration jobs will be submitted to the web service instead of SLURM.
The application will end 30 minutes after the last submission of the job. 

The calibration jobs in SLURM-only mode are not real processing, instead, they are the same as normal jobs, i.e., executing the $sleep$ command.
With the configuration, the calibration jobs in SLURM-only mode will take five nodes for 240 minutes, which is close to the real case where the test data is processed using five nodes.

Besides, there is a monitor process that records the status of the cluster.
The monitor records the resource occupation of different jobs (e.g., calibration, normal, others, and pending).
The simulation program logs the submission time of the calibration jobs, and the calibration application itself logs the finishing of calibration jobs.
Thus, the waiting time can be calculated.
Additionally, the simulation application also records two internal values: the $miniNodes$ and $leftTasks$, which is the number of pending and running tasks.
Monitoring these two values helps us to validate whether the scaling policy acts as expected.
The mapping between $miniNodes$ and $leftTasks$ can be formulized as Eq. \eqref{con:mininode}.
\begin{equation}
    miniNodes=\left\{
        \begin{aligned}
        2 &      & {leftTasks      <      5}\\
        4 &      & {5< leftTasks      <      20}\\
        leftTasks/20+3 &      & {20< leftTasks      <      100}\\
        10 &      & {else}\\
        \end{aligned}
        \right.\label{con:mininode}
\end{equation}
The mapping function is also part of the performance parameters where we can configure the proposed system how greedy on resources for the calibration use case.
 
\subsection{Scenario 1: Not heavily loaded cluster }
First, we consider the scenario of the Non-busy cluster. In this case, most of the time, the cluster is not fully utilized due to a lack of jobs.
The synthetic workload is described in \ref{submit:1}, which consists of 19 jobs, including four calibration jobs and 15 other jobs. 
With this workload, the test takes about  11 hours, and during the test, in the SLURM-only mode, the resource usage is not optimal in Fig. \ref{fig:nonBusyMPIUTI} this idle resource is shown by the white area under the horizontal line.
The cluster has 24 working modes, and during the experiments, the average resource occupation is 19.79(nodes), which means the overall resource utilization rate is 82.47\%.
The calibration jobs take 5.81 nodes on average, and other users take 13.99 nodes.
\begin{figure}
    \centering
    \begin{minipage}{.48\textwidth}
      \centering
      \includegraphics[width=1\linewidth]{4_analysis&results/figures/MPIUtiNonBusy.png}
      \caption[Resource utilization on SLURM-only mode ,non busy case]{{\small\textbf{Resource utilization on SLURM-only mode ,non busy case} - The overall resource utilization is 82.47\%}}
      \label{fig:nonBusyMPIUTI}
    \end{minipage} 
    \begin{minipage}{.48\textwidth}
      \centering
      \includegraphics[width=1\linewidth]{4_analysis&results/figures/MPIGanttNonBusy.png}
      \caption[Gantt chart of calibration jobs]{{\small\textbf{Gantt chart of calibration jobs} - Benefit from adequate resources jobs start once they are submitted, the vertical line is the time simulation ends }}
      \label{fig:nonBusyMPIgantt}
    \end{minipage}
\end{figure}
%\figuremacroW{4_analysis&results/figures/MPIUtiNonBusy.png}{Resource utilization on MPI mode ,non busy case}{The overall resource utilization is 82.47\%}{0.85}
According to Fig. \ref{fig:nonBusyMPIgantt} and \ref{submit:1}, each job almost starts immediately after submitted due to the enough resources.
%\figuremacroW{4_analysis&results/figures/MPIGanttNonBusy.png}{Gantt chart of calibration jobs}{ Benefit from adequate resources jobs start once they are submitted}{0.85}
\begin{figure}
    \centering
    \begin{minipage}{.48\textwidth}
      \centering
      \includegraphics[width=1\linewidth]{4_analysis&results/figures/ScaleUtiNonBusy.png}
      \caption[Resource utilization after introducing this system ,non busy case]{{\small\textbf{Resource utilization after introducing this system ,non busy case} - The overall resource utilization is 99.83\%}}
      \label{fig:nonBusyScaleUTI}
    \end{minipage} 
    \begin{minipage}{.48\textwidth}
      \centering
      \includegraphics[width=1\linewidth]{4_analysis&results/figures/ScaleGanttNonBusy.png}
      \caption[Gantt chart of calibration jobs]{{\small\textbf{Gantt chart of calibration jobs} - Job 47 is accelerated due to extra resources }}
      \label{fig:nonBusyScalegantt}
    \end{minipage}
\end{figure}
In the second part of the test, we switch to scale mode. Fig. \ref{fig:nonBusyScaleUTI} comparing to Fig. \ref{fig:nonBusyMPIUTI}, the white area under the horizontal line shows the idle resources has completely disappeared.
The nominal overall resource utilization is  99.83\%. The average resource occupation of normal jobs is 13.97 nodes, and for calibration applications, it increases to 9.99 nodes.
In terms of waiting time, Fig. \ref{fig:nonBusyScalegantt} shows that jobs cost less time to finish the same task.
The waiting time is reduced from 240 minutes to 121 minutes.

Note that the waiting time reduction of Job 45 is not completely due to the auto-scaling.
As within the proposed system, the execution order follows First come, first served(FCFS).
Job 45 takes 10 nodes which is double the number in SLURM mode. 
Job 46 waits for Job 45 to finish instead of processing data at the same time.  

This simulation shows in a non-busy scenario, the system can help to accelerate jobs for every user.

\subsection{Scenario 2: Heavily loaded cluster}

In a non-busy scenario, the overall resource utilization is not high.
The proposed user-level resource management system initially aims to solve the problems of the limitations of the backfilling scheduling policy.
Therefore, we create a heavy synthetic workload to study the resource utilization taking into account the backfill scheduling algorithm.
There will be 24 jobs submitted to the cluster/system which includes 8 calibration jobs and 16 normal jobs in this busy scenario. The workload is listed in \ref{submit:2}.
\begin{figure}
    \centering
    \begin{minipage}{.48\textwidth}
      \centering
      \includegraphics[width=1\linewidth]{4_analysis&results/figures/MPIUtiBusy.png}
      \caption[Resource utilization on SLURM-only mode, busy case]{{\small\textbf{Resource utilization on SLURM-only mode, busy case} - The overall resource utilization is 90.57\%}}
      \label{fig:BusyMPIUTI}
    \end{minipage} 
    \begin{minipage}{.48\textwidth}
      \centering
      \includegraphics[width=1\linewidth]{4_analysis&results/figures/MPIGanttBusy.png}
      \caption[Gantt chart of calibration jobs]{{\small\textbf{Gantt chart of calibration jobs} - Jobs need to take a while waiting for resources}}
      \label{fig:BusyMPIgantt}
    \end{minipage}
\end{figure}
It can be seen from Fig. \ref{fig:BusyMPIUTI}, the utilization can be measured, and on average 90.57\% of the resources are occupied.
However, during the test, part of the resources is idle due to the limit of scheduling policy, even though there are plenty of jobs on the pending.
Fig. \ref{fig:BusyMPIgantt} shows that that calibration jobs take a longer time to complete compared to scenario 1 (non-busy scenario).



\begin{figure}[h]
    \centering
    \begin{minipage}{.48\textwidth}
      \centering
      \includegraphics[width=1\linewidth]{4_analysis&results/figures/ScaleUtiBusy.png}
      \caption[Resource utilization after introducing this system ,busy case]{{\small\textbf{Resource utilization after introducing this system ,busy case} - The overall resource utilization is 99.86\%}}
      \label{fig:BusyScaleUTI}
    \end{minipage} 
    \begin{minipage}{.48\textwidth}
      \centering
      \includegraphics[width=1\linewidth]{4_analysis&results/figures/ScaleGanttBusy.png}
      \caption[Gantt chart of calibration jobs]{{\small\textbf{Gantt chart of calibration jobs} - Users need less time for waiting}}
      \label{fig:BusyScalegantt}
    \end{minipage}
\end{figure}

\begin{table}[]
    \centering
    \begin{tabular}{ccccccccc}
    \hline
    Job ID         & 0      & 1      & 2      & 3      & 4      & 5      & 6      & 7     \\
    \hline
    SLURM-only mode(Min)   & 240.01 & 240.01 & 402.60 & 335.53 & 414.91 & 316.52 & 206.51 & 49.17 \\
    Scale mode(Min) & 118.15 & 251.99 & 191.73 & 328.04 & 332.95 & 281.57 & 206.51 & 49.18
    \end{tabular}
    \caption{Waiting time of jobs comparison on SLURM-only mode and Scale mode}
    \label{tab:waittimecomparison}
\end{table}
In the second part of the test we switched to scale mode, the overall nominal utilization climbs to 99.86\%. 
In Fig. \ref{fig:BusyScaleUTI}, the pink curve represents the number of pending and running tasks.
During the entire simulation, the curve stays at a high position, which indicates there is no wasted resource taken by the system.
Therefore, the nominal resource utilization is close to the real resource utilization of physical computation capability.

The calibration applications, by comparing the waiting time of calibration jobs, the extra resources also benefit the calibration jobs.
It can be observed from  Table. \ref{tab:waittimecomparison} and Fig. \ref{fig:BusyScalegantt} that, except for the jobs of 0 and 1, they are affected by the queue strategy and the jobs of 6 and 7, which are not finished at the end of the simulation. 
All other jobs, the jobs from 2-5, take less time after the introduction of this system.

Besides the scenario experiments, we also performed a test in the production environment hoping to measure the resource utilization under different workload patterns.
The experiment was performed in the DAS5-Leiden site with full 23 available working nodes in seven days. 
But unfortunately, due to the lack of other users, the workload is lighter than our non-busy scenario. 
Therefore, we were not able to test the performance in different environments.
% \subsection{Resource utilization testing on real environment}
% The simulations above are constructed manually.
% To evaluate the performance of resource management in real scenarios, a 7-day experiment was performed, which started at 0:00 AM 22ed of August and ends at 11.59 PM 28th of August in 2020.
% The experiment was performed in the DAS5-Leiden site with full 23 available working nodes. 
% The resource occupation is displayed in Fig. \ref{4_analysis&results/figures/7days.png}  
% \figuremacroW{4_analysis&results/figures/7days.png}{7 days experiment with other users come and go}{The overall resource utilization is 99.99\%}{1}
% Unfortunately, due to the summer session, only two other users submitted the jobs during the experiment time.
% The overall nominal resource utilization achieves 99.99\%.
% ---------------------------------------------------------------------------
% ----------------------- end of thesis sub-document ------------------------
% ---------------------------------------------------------------------------