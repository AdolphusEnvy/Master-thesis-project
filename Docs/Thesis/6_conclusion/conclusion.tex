% this file is called up by thesis.tex
% content in this file will be fed into the main document

\chapter{Conclusion} % top level followed by section, subsection


% ----------------------- paths to graphics ------------------------

% change according to folder and file names


% ----------------------- contents from here ------------------------
% 


%-----------------------------overview-----------------------------------

In this paper, we proposed a self-adjusted auto-provision system which aims to achieve as high resource utilization as possible via making adaption to the cluster backfill scheduling algorithm. 
The system tries to take all the idle resources on the one hand, and be friendly to other users on the other hand. 
In general, with more resources putting in use, the cluster is available to process more jobs for the long term.

We choose the LOFAR image calibration process as our test use case, and the calibration process is implemented in a distributed form. 
The overall goal is to increase resource utilization and, at the same time, speed up the calibration jobs. 
After the introduction of the system, the results show that the calibration jobs get accelerated by more resources, and the time which end-users have to wait is shortened. 
Besides, the resources allocated to other users’ jobs do not change a lot. 
In general, there is no stakeholder receives bad effect after introducing this system into our test cases.

For future work, the GPU and job array should be supported because they are very common to the current clusters. 
Since the GPU features are cluster- specified, it also needs to provide an interface to configure this information. 
The core algorithm has rooms for improvement as well. 
For instance, to reduce the scheduling time of SLURM, we should enable this system to request resources with a larger size if needed.

The results of experiments and simulations show that the cluster can reach the nominal resource utilization of 99.8\% , and users save waiting time ranging from  5\%(busy workload) to 50\%(idle workload) compared with the baseline(MPI). 
In general, this system has achieved its objective to increase the resource utilization and accelerate parallel jobs.

% ---------------------------------------------------------------------------
% ----------------------- end of thesis sub-document ------------------------
% ---------------------------------------------------------------------------