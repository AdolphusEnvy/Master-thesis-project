% this file is called up by thesis.tex
% content in this file will be fed into the main document

\chapter{Conclusion} % top level followed by section, subsection


% ----------------------- paths to graphics ------------------------

% change according to folder and file names


% ----------------------- contents from here ------------------------
% 


%-----------------------------overview-----------------------------------

In this report, we proposed a self-adjusted auto-provision system that aims to achieve as high resource utilization as possible through making adapting to the cluster backfill scheduling algorithm. 
The system tries to harvest all the idle resources, and be friendly to other users in the meantime. 
In general, with more resources putting in use, the cluster is available to process more jobs.

We choose the LOFAR image calibration process as our test use case, and the calibration process is implemented in a distributed form. 
The overall goal is to increase resource utilization and, at the same time, speed up the calibration jobs. 
After the adaption of the system, the results show that the calibration jobs get accelerated by more resources assigned, and the time which end-users have to wait is shortened. 
Besides, the resources allocated to other users’ jobs do not change a lot. 
In general, no user receives bad effects after introducing this system in our test cases.

For future work, the GPU and job array should be supported because they are very common to modern clusters. 
Since the GPU features are cluster-specified, we also need to provide an interface to collect the related information. 
Additionally, the core algorithm has room for improvement as well. 
For instance, to reduce the scheduling time of SLURM, we should enable this system to request resources with a larger size when needed.

The results of experiments and simulations show that the cluster can reach the nominal resource utilization of 99.8\%, and users save waiting time ranging from  5\%(busy workload) to 50\%(idle workload) compared with the baseline(SLURM-only). 
In general, this system has achieved its objective to increase resource utilization and accelerate parallel jobs.

% ---------------------------------------------------------------------------
% ----------------------- end of thesis sub-document ------------------------
% ---------------------------------------------------------------------------